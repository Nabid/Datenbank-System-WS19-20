
\documentclass[a4paper]{article}
\usepackage[utf8]{inputenc}

\usepackage[english,german]{babel} 
\usepackage[utf8]{inputenc}

\usepackage{alltt}
\usepackage{amsmath}
\usepackage{amssymb}
\usepackage{amsthm}
\usepackage{color}
\usepackage{enumitem}
\usepackage{epsfig}
\usepackage{fancyhdr}
\usepackage{float}
\usepackage{framed}
\usepackage{graphicx} 
\usepackage{graphics}
\usepackage{hyperref}
\usepackage{listings}
\usepackage{multirow}
\usepackage{tabularx}
\usepackage{textcomp}
\usepackage{tikz}
\usepackage{url}
\usepackage{vmargin}
\usepackage{xspace}
\usepackage{comment}
\usetikzlibrary{calc,trees,positioning,arrows,chains,shapes.geometric,%
    decorations.pathreplacing,decorations.pathmorphing,shapes,%
    matrix,shapes.symbols,topaths,matrix}
\newcommand{\question}[2][0]{\section{{#2} \hfill ({#1} P.)}}
\setpapersize{A4}
\setmargins{2.5cm}{2.0cm}% % linker & oberer Rand
         {16cm}{22cm}%   % Textbreite und -hoehe
           {48pt}{36pt}%   % Kopfzeilenhoehe und -abstand
           {0pt}{30pt}%    % \footheight (egal) und Fusszeilenabstand

\frenchspacing
\pagestyle{fancy}
\sloppy

\newcommand{\hide}[1]{}

\markright{Kopfzeile}


\tikzset{
>=stealth',
  punktchain/.style={
    rectangle,
    rounded corners,
    % fill=black!10,
    draw=black, very thick,
    text width=10em,
    minimum height=3em,
    text centered,
    on chain},
  line/.style={draw, thick, <-},
  element/.style={
    tape,
    top color=white,
    bottom color=blue!50!black!60!,
    minimum width=8em,
    draw=blue!40!black!90, very thick,
    text width=10em,
    minimum height=3.5em,
    text centered,
    on chain},
  every join/.style={->, thick,shorten >=1pt},
  decoration={brace},
  tuborg/.style={decorate},
  tubnode/.style={midway, right=2pt},
}


\setlength{\parindent}{0pt}
\setlength{\parskip}{5pt}
\fboxsep1.5mm


\lstdefinestyle{mystyle}{
    backgroundcolor=\color{white},   
    commentstyle=\color{codegray},
    keywordstyle=\bf \ttfamily \color{codepurple},
    numberstyle=\tiny\color{codegray},
    stringstyle=\color{codegreen},
    basicstyle=\footnotesize,
    breakatwhitespace=false,         
    breaklines=true,                 
    captionpos=b,                    
    keepspaces=true,                 
    numbers=left,                    
    numbersep=5pt,                  
    showspaces=false,                
    showstringspaces=false,
    showtabs=false,                  
    tabsize=8,
    keepspaces,
    extendedchars=true, 
      upquote=true,
    columns=fixed,
    showstringspaces=false,
    extendedchars=true,
    breaklines=true,
    frame=single,
    showspaces=false,
    showstringspaces=false,
    rulecolor=\color{white},
}

\lstdefinelanguage{sql}[]{}{
        %tag=[s]<>,      % =*: also apply styles within tag, =**: cumulate styles
        morekeywords={sql, VIEW, AS, FROM, SELECT, WHERE, FUNCTION, BOOLEAN, RETURNS, DETERMINISTIC, RETURN, REFERENCES, WITH, SEQUENCE, TRUNCATE, START, CREATE, AS, LANGUAGE, FUNCTION, CURSOR, PREPARE, OPEN, USING, CLOSE, DECLARE, END, BEGIN, EXEC, SQL, CONNECT TO, DISCONNECT, COMMIT, LOOP, IF, THEN, ELSE, WHILE, BREAK, EXIT, INSERT, INTO, VALUES, UPDATE, SET, TABLE, PRIMARY, KEY, AND, UNION, ALL, JOIN, ON, GROUP, BY, MATERIALIZED, INT, DATE, COUNT, ORDER, OVER, PARTITION, ASC, DESC, VARCHAR, NOT, NULL, PRIMARY, KEY, DECIMAL, SUM, AVG, ROWS, BETWEEN, PRECEDING, CURRENT, ROW, INSTEAD, TRIGGER, OF, FOR, EACH, EXECUTE, PROCEDURE, DISTINCT, HAVING, LIMIT},
morestring=[s]{'}{'},
morecomment=[l]{--}
        %sensitive=false
}

\lstset{style=mystyle,numbers=none,basicstyle=\ttfamily,upquote=true}

 
\definecolor{codegreen}{rgb}{0,0.6,0}
\definecolor{codegray}{rgb}{0.5,0.5,0.5}
\definecolor{codepurple}{rgb}{0.38,0,0.72}
\definecolor{backcolour}{rgb}{0.95,0.95,0.92}
\definecolor{backcolourSingleCode}{rgb}{0.95,0.95,0.92}

\newcommand{\subtitle}{\textbf{Exercise 4}}
\newcommand{\outdate}{19.11.2019}
\newcommand{\duedate}{25.11.2018 16:00 CET}



\begin{document}

\lhead{\begin{tabular}{l}
{\bf Database Systems WS 2019/20}\\
{\bf \subtitle: Distributed \outdate, Due \duedate}\\
{Submitted by Priom Biswas, Md Nabid Imteaj }
\end{tabular}
}
\rhead{}


\question[1]{Wavelets, Sketches}

\begin{enumerate}
\item
  Calculate the wavelet transform of the following cumulative distribution over the numbers 0 to 7.
  Also calculate the Mean-Squared-Error (MSE) if you omit all coefficients $\leq 3$.

  $$[2, 6, 8, 12, 20, 25, 26, 29]$$
 
 
  

\item Given the following Table with elements $x \in X$ and the hash values for the hash function $h: X \rightarrow [0,1]$.\\

\begin{center}
\begin{minipage}{0.2\columnwidth}
\begin{tabular}{|l|r|} \hline
$x$  & $h(x)$ \\ \hline \hline
Toyota & 0.05 \\ \hline
Ford & 0.489 \\ \hline
Bughatti & 0.678 \\ \hline
BMW & 0.281 \\ \hline
Audi & 0.81 \\ \hline
\end{tabular}
\end{minipage}
\begin{minipage}{0.3\columnwidth}
\begin{tabular}{|l|r|} \hline
$x$  & $h(x)$ \\ \hline \hline
Mercedes & 0.15 \\ \hline
Dodge & 0.565 \\ \hline
Porsche & 0.579 \\ \hline
Nissan & 0.395 \\ \hline
Hyundai & 0.91 \\ \hline
\end{tabular}
\end{minipage}
\end{center}

Using the KMV Sketch algorithm for $ k \in {1, 2, 3, 4, 5} $, calculate the estimate as well as the absolute error. Will the error get smaller with a larger $ k $? How does this generally relate to the KMV Sketch?

 

\end{enumerate}

\newpage


\question[1]{Statistics in PostgreSQL}
To improve the query optimization, PostgreSQL keeps track of various statistics about every column in every table.
Consider the following statistics about the {\it lineitem} table from the TPC-H database:


\verb+SELECT * FROM pg_stats WHERE tablename = 'lineitem'+

\begin{table}[H]
\begin{tabular}{|l|l|l|l|l|}
\hline
\textbf{attname} & \textbf{null\_frac} & \textbf{n\_distinct} & \textbf{most\_common\_vals} & \textbf{most\_common\_freqs} \\ \hline

l\_linenumber & 0 & 7 & \{1,2,3,4,5,6,7\} & \{0.25,0.21,0.17,0.14,0.10,0.07,0.03\} \\ \hline
l\_orderkey & 0 & 419664 & \multicolumn{1}{|p{4cm}|}{ \{1703939,3644579,507137,\newline 703172,712326,770882,\newline 971014\} }
                         & \multicolumn{1}{|p{4cm}|}{ \{0.000133333,0.000133333,0.0001,\newline 0.0001,0.0001,0.0001,\newline 0.0001\} } \\ \hline
l\_extendedprice  & 0 & -0.11964 & \multicolumn{1}{|p{4cm}|}{ \{13747.3,25165.2,\newline 26677.8,32274.0,\newline 33265.3,35938.8,50370.3 \} } &
                                   \multicolumn{1}{|p{4cm}|}{ \{0.0001,0.0001,0.0001,0.0001,\newline 0.0001,0.0001,0.0001 \} } \\ \hline

\end{tabular}
\end{table}
\begin{enumerate}
\item What does each column of \emph{pg\_stats} represent? 

\item Which of the following queries benefit from these statistics? The total size of the table is $6\,001\,215$.
\begin{itemize}[label={}]
\setlength{\itemsep}{0cm}
\item $\Box$ \verb+SELECT * FROM lineitem+ 

\item $\Box$ \verb+SELECT * FROM lineitem WHERE l_orderkey IS NOT NULL+ 

\item $\Box$ \verb+SELECT * FROM lineitem WHERE l_orderkey = 406631+ 

\item $\Box$ \verb+SELECT * FROM lineitem WHERE l_extendedprice = 32274.0+

\item $\Box$ \verb+SELECT * FROM lineitem WHERE l_extendedprice <= 9500.0+ 

\end{itemize}
\item Use the statistics to estimate the result size of each query in part (b).

\end{enumerate}
In the table below an additional column of the \emph{pg\_stats} table is shown. It shows ''A list of values that divide the column's values into groups of approximately equal population. The values in \emph{most\_common\_vals}, if present, are omitted from this histogram calculation.'' -- \url{https://www.postgresql.org/docs/10/static/view-pg-stats.html}

\begin{center}
\begin{tabular}{|l|l|l|}
\hline
\textbf{attname} & \textbf{histogram\_bounds}             \\ \hline
l\_extendedprice & \multicolumn{1}{|p{10cm}|}{  \{906.00,1528.60,2062.06,3035.16,3639.44,4444.04,5265.52,5920.60,\newline 6751.85,7518.63,8252.80,9032.32,9777.68,10540.68,11267.46,\newline 11921.76,12652.00,13413.40,14078.28 (...) \} }
 \\ \hline
\end{tabular}
\end{center}


\begin{enumerate}
\setcounter{enumi}{3}
\item Decide for which queries in (b) this additional column is useful and estimate the result set size.

\end{enumerate}

\newpage
\question[1]{Join ordering}

Given the following query on the TPC-H database:

\begin{lstlisting}[language=sql]
SELECT *
FROM orders o, customer c, lineitem l, nation n,
WHERE o.o_custkey = c.c_custkey
 AND l.l_orderkey = o.o_orderkey AND c.c_nationkey = n.n_nationkey
\end{lstlisting}

\begin{enumerate}
\item
 Draw the query graph.
 

\item
 Calculate the join selectivities. 

\item
 Calculate the cost $C_{out}$ for all possible join trees without cross products.
\end{enumerate}

\question[1]{Join Ordering}

\begin{enumerate}
\item
  {\bf Left-deep trees vs. Bushy trees}\\
  Given the following relations: $R_1$, $R_2$, $R_3$ and $R_4$,
  with $|R_1|=100$, $|R_2|=50$, $|R_3|=200$, $|R_4|=10$, and the following join selectivities:
  $j_{1,2}= 0.04$, $j_{2,3}= 0.10$, $j_{2,4}= 0.05$. 

  \begin{enumerate}
    \item Draw the query graph.
      
    \item List all left-deep trees without cross products. 
      
      
    \item Calculate the $C_{out}$ cost for the join orderings created in (ii) where $R_3$ is joined last (on the top of the tree).
    

  \end{enumerate}

  \item   Is there a bushy tree (that is not a linear tree), without cross products, that has lower cost?
      

\end{enumerate}

\question[1]{Join Ordering Algorithms}

Write a program which compares the different join ordering algorithms, presented in the lecture.
You may use any language you like.
You have two write one method for each greedy algorithm plus two methods that find the best and worst possible join ordering.
For the last two functions you have to find all possible orderings.
Use the $C_{out}$ cost to compare the algorithms.

You may use the Java template provided in OLAT.
The template provides all required boilerplate code and instructions on how to use it.

Use your program to generate plans for the following relations: 
$|R_1| = 200, |R_2| = 300, |R_3| = 20, |R_4| = 140$ with the selectivities: $j_{1,3} = 1/100$, $j_{2,3} = 1/50$, $j_{3,4} = 1/5$
Cross products are not allowed.


Submit the created plans with their $C_{out}$ costs and the code that created them.

HINT: The best and worst plans range somewhere between $(8\,000,11\,000)$.
 
\end{document}
